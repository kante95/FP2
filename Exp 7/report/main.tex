\documentclass[a4paper,10pt]{article} 

\usepackage[utf8]{inputenc} 
%\usepackage[T1]{fontenc}

\usepackage{textcomp}           % Extra Symbole (Grad Celsius etc.)
\usepackage{amssymb,amsmath}    % Schöne Formeln (AMS = American Mathematical Society)
\usepackage{graphicx}           % Bilder und Seitenränder
\usepackage{subcaption}			% captions for subfigures
\usepackage{booktabs}           % Schönere Tabellen
\usepackage{colortbl}           % Farbige Tabellen

%\usepackage{tcolorbox}			% schöne bunte Boxen
\usepackage{mathtools}			% \mathclap für ordentliche \underbrace-			environments
\usepackage{geometry}			% Pagelayout mit \newgeometry, \restoregeometry
\usepackage{float}
\usepackage{wrapfig}
\usepackage{enumitem}
\usepackage{float}
\usepackage{braket}
\usepackage{caption}
\usepackage{pst-optexp}
\usepackage{auto-pst-pdf}

\graphicspath{{./img/}}


\bibliographystyle{unsrtnat}

\renewcommand{\k}{\mathbf{k}}
\begin{document}
\begin{titlepage}
 \begin{center}
	\Large{Advanced laboratory class 2}
	\end{center}
	\begin{center}
	 \LARGE{\textbf{FP2 - Nonlinear Optics - Second Harmonic Generation}}
	\end{center}
	
	\begin{center}
	
	\large Marco \textsc{Canteri} \\
	marco.canteri@student.uibk.ac.at
	\end{center}
	
	\begin{center}
	\vspace{1cm}
	Innsbruck, \today
	\vspace{2cm}
	\end{center}
	
	\begin{center}
	\includegraphics[scale=0.4]{img/uibk} 
	\end{center}

\end{titlepage}
\begin{abstract}
In this work we generated ultraviolet light at around $317$ nm from a laser beam of $633$ nm exploiting Second Harmonic Generation (SHG), a second order non linear effect of a pottassium
dihydrogen phospahte (KDP) crystal. We measured the power of the red laser as a function of the angle of a polarizer, then we studied the efficiency of the SHG with respect to the crystal angle.
\end{abstract}
\section{Introduction}
Non linear optics bla bla bla
\section{Experiment setup}
\begin{figure}[H]
\centering
\begin{pspicture}[showgrid=false](8,2)(-3,-2)
\pnode(1,1){A}\pnode(5,1){B}\pnode(5,-1){C}\pnode(1,-1){D}
\optsource[position=start,innerlabel](A)(B){\parbox{1.5cm}{\centering He:Ne\\ 633\,nm}}
\optretplate[position=0.2](A)(B){$\nicefrac{\lambda}{2}$}
\optplate[position=0.6](A)(B){pol}
\mirror(A)(B)(C)
\mirror(B)(C)(D)
\optbox[position=start,innerlabel](D)(C){PMT}
\optbox[innerlabel,position=.5,optboxwidth=0.9](D)(C){KDP}
\lens[position=.35,lensheight=0.6](D)(C){}
\lens[position=.65,lensheight=0.6](D)(C){}
\pinhole[position=.95,outerheight=.7](C)(D){}
\drawbeam[linecolor=red]{1}{2}{3}{4}{5}{9}{7}
\drawbeam[linecolor=blue]{7}{8}{6}
\end{pspicture}
\caption{Experiment setup. A red laser is pumped into a KDP crystal to generate ultraviolet light at 317 nm (showed in blue in this figure) detected with a photomultiplier}
\end{figure}
\section{Measurements and analysis}

 \begin{thebibliography}{99}

  \bibitem{bellpaper}
     \textsc{J. Bell}, \textit{On the Einstein Podolsky Rosen paradox}, Physics, 1 (1964), pp. 195–200.

  \bibitem{inequality}
   \textsc{J. F. Clauser, M. A. Horne, A. Shimony und R. A. Holt}, \textit{Proposed experiment to
test local hidden-variable theories}, Phys. Rev. Lett., 23 (1969), pp. 880–884.

\bibitem{skriptum}
Fortgeschrittenenpraktikum 2, \textit{Entanglement and Bell’s inequality}. \textsc{Gregor Weihs, Kaisa Laiho, Harishankar Jayakumar}. WS 2015/16
\end{thebibliography}
\end{document}
